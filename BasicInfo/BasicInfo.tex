\documentclass[a4paper, 12pt]{article}

\usepackage{enumerate}
\usepackage{hyperref}
\hypersetup{
	colorlinks=true,
	linkcolor=blue,
	filecolor=blue,
	urlcolor=blue,
	citecolor=blue,
}
\usepackage{amsmath}
\usepackage{amsthm}
\usepackage{amssymb}
\usepackage[margin=3cm]{geometry}
\usepackage{mathpazo}
\usepackage{url}
\usepackage[labelformat=simple]{subcaption}
\usepackage{tikz}
\usepackage{pgf}
\usepackage{longtable}
\usepackage{multirow}
\usepackage{graphicx}

\begin{document}
\pagestyle{empty}

\begin{center}
{\Large Computing and Data Science -- The Python Part (CS102-4)} 

\vspace{0.25cm}

{\large Joshua Maglione}

\vspace{0.25cm}

Semester 2 (2026)
\end{center}

\vspace{0.5cm}

\begin{description}
    \item[Module information:] \hfill
    \begin{description}
      \item[Meeting Times:] \hfill
      \begin{itemize}
        \item Tuesdays 2:00pm -- 2:50pm (\href{https://clients.mapsindoors.com/nuigalwayweb/9167eab0dc78437c93c76b57/details/016a4184bae44d70853a02e5}{CSB-1007})
        \item Wednesdays 2:00pm -- 2:50pm (\href{https://clients.mapsindoors.com/nuigalwayweb/9167eab0dc78437c93c76b57/details/22185f1db04149f3ab4bac48}{AC202})
        \item Thursdays 9:00am -- 9:50am (\href{https://clients.mapsindoors.com/nuigalwayweb/9167eab0dc78437c93c76b57/details/76ccb06aa36c4758823836e7}{Tyndall})
      \end{itemize}
      \item[Contact:] \url{joshua.maglione@universityofgalway.ie},
      \item[Website:] \href{https://universityofgalway.instructure.com/}{\textsf{Canvas}} and \url{https://joshmaglione.com/2025CS102-4.html}.
    \end{description} 

    \noindent The Python part of this module comprises the first 12 lectures. 

    \vspace{1cm}
    \item[Topics:] We will cover the basics of four key topics in this module:
    \begin{enumerate} 
      \item Scientific Computing,
      \item Data Analysis,
      \item Data Visualization, and
      \item Machine Learning.
    \end{enumerate} 
    We do this by exploring four ubiquitous Python packages: \texttt{NumPy},
    \texttt{pandas}, \texttt{Matplotlib}, and \texttt{scikit-learn}.
    \vspace{1cm}
    \item[Assessment:] There will be four ???
    \vspace{1cm}
    \item[Reading:] We will go through \textit{Python Data Science Handbook} by Jake VanderPlas. The book is available online:
    \href{https://jakevdp.github.io/PythonDataScienceHandbook/}{Python Data Science Handbook}. 
    
    As a secondary source, one might consider \textit{IPython Interactive Computing and Visualization Cookbook, 2nd edition} by Cyrille Rossant. The book is available online:
    \href{https://ipython-books.github.io/}{IPython Cookbook}.
\end{description}



\end{document}
