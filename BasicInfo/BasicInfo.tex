\documentclass[a4paper, 12pt]{article}

\usepackage{enumerate}
\usepackage{hyperref}
\hypersetup{
	colorlinks=true,
	linkcolor=blue,
	filecolor=blue,
	urlcolor=blue,
	citecolor=blue,
}
\usepackage{amsmath}
\usepackage{amsthm}
\usepackage{amssymb}
\usepackage[margin=3cm]{geometry}
\usepackage{mathpazo}
\usepackage{url}
\usepackage[labelformat=simple]{subcaption}
\usepackage{tikz}
\usepackage{pgf}
\usepackage{longtable}
\usepackage{multirow}
\usepackage{graphicx}

\begin{document}
\pagestyle{empty}

\begin{center}
{\Large Computer Science -- Further Computing (CS102-4)} 

\vspace{0.25cm}

{\large Joshua Maglione}

\vspace{0.25cm}

Semester 2 (2024)
\end{center}

\vspace{0.5cm}

\begin{description}
    \item[Module information:] \hfill
    \begin{description}
      \item[Meeting Times:] Tuesdays 2:00pm -- 2:50pm,
      \item[Room:] Computer Science Building 202,
      \item[Contact:] \url{joshua.maglione@universityofgalway.ie},
      \item[Website:] \href{https://universityofgalway.instructure.com/}{\textsf{Canvas}} and \url{https://joshmaglione.com/2024CS102-4.html}.
    \end{description} 
    \vspace{1cm}
    \item[Topics:] We will cover the basics of four key topics in this module:
    \begin{enumerate} 
      \item Scientific Computing,
      \item Data Analysis,
      \item Data Visualization, and
      \item Machine Learning.
    \end{enumerate} 
    We do this by exploring four ubiquitous Python packages: \texttt{NumPy},
    \texttt{pandas}, \texttt{Matplotlib}, and \texttt{scikit-learn}.
    \vspace{1cm}
    \item[Assessment:] There will not be any assessment for CS102-4.
\end{description}



\end{document}
